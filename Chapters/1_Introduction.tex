%!TEX root = ../MainBody.tex

% 第一章
\chapter{绪论}% 使用\cite{}命令引用数据库中文献
引言(或绪论)应包括选题的背景和意义,国内外相关研究成果与进展述评,本论文所要解决的科学与技术问题、所运用的主要理论和方法、基本思路和论文结构等。引言应独立成章,用足够的文字叙述,不与摘要雷同。要求实事求是,不夸大、缩小前人的工作和自己的工作。
\section{选题背景及意义}
要论述清楚为什么选择这个题目来研究,即阐述该研究对学科发展的贡献、对国计民生的理论与现实意义等。四川大学标志如\cref{fig:test}所示。
%插入图片示例
\begin{figure}[!htb]
    \centering
    \begin{subfigure}[b]{0.35\textwidth}
    \includegraphics[width=\textwidth]{./Template/Components/Images/SCU_TITLE}
    \bicaption{红色标志}{Red logo}
    \label{fig:sub_a}
    \end{subfigure}%
    \qquad
    \begin{subfigure}[b]{0.35\textwidth}
        \includegraphics[width=\textwidth]{./Template/Components/Images/SCU_TITLE_BW}
        \bicaption{黑色标志}{Black logo}
        \label{fig:sub_b}
    \end{subfigure}%
    \bicaption{四川大学标志}{Logo of Sichuan University}
    \label{fig:test}
\end{figure}

\section{文献综述}
要对本研究主题范围内的文献进行详尽的综合述评,“述”的同时一定要有“评”,指出现有研究状态,仍存在哪些尚待解决的问题,讲出自己的研究有哪些探索性内容。曹敏\cite{曹敏GB,陈浩元2015gb}对修订后的GB/T 7714—2015《信息与文献参考文献著录规则》与上一版本GB/T 7714—2005《文后参考文献著录规则》的差异进行了分析,帮助读者尽快了解参考文献标准最新修订的内容。
\section{研究目的与实施方案}
讲述本论文运用的主要理论与研究方法、基本思路及论文的结构安排等。

